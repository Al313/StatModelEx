% Options for packages loaded elsewhere
\PassOptionsToPackage{unicode}{hyperref}
\PassOptionsToPackage{hyphens}{url}
%
\documentclass[
]{article}
\usepackage{amsmath,amssymb}
\usepackage{iftex}
\ifPDFTeX
  \usepackage[T1]{fontenc}
  \usepackage[utf8]{inputenc}
  \usepackage{textcomp} % provide euro and other symbols
\else % if luatex or xetex
  \usepackage{unicode-math} % this also loads fontspec
  \defaultfontfeatures{Scale=MatchLowercase}
  \defaultfontfeatures[\rmfamily]{Ligatures=TeX,Scale=1}
\fi
\usepackage{lmodern}
\ifPDFTeX\else
  % xetex/luatex font selection
\fi
% Use upquote if available, for straight quotes in verbatim environments
\IfFileExists{upquote.sty}{\usepackage{upquote}}{}
\IfFileExists{microtype.sty}{% use microtype if available
  \usepackage[]{microtype}
  \UseMicrotypeSet[protrusion]{basicmath} % disable protrusion for tt fonts
}{}
\makeatletter
\@ifundefined{KOMAClassName}{% if non-KOMA class
  \IfFileExists{parskip.sty}{%
    \usepackage{parskip}
  }{% else
    \setlength{\parindent}{0pt}
    \setlength{\parskip}{6pt plus 2pt minus 1pt}}
}{% if KOMA class
  \KOMAoptions{parskip=half}}
\makeatother
\usepackage{xcolor}
\usepackage[margin=1in]{geometry}
\usepackage{color}
\usepackage{fancyvrb}
\newcommand{\VerbBar}{|}
\newcommand{\VERB}{\Verb[commandchars=\\\{\}]}
\DefineVerbatimEnvironment{Highlighting}{Verbatim}{commandchars=\\\{\}}
% Add ',fontsize=\small' for more characters per line
\usepackage{framed}
\definecolor{shadecolor}{RGB}{248,248,248}
\newenvironment{Shaded}{\begin{snugshade}}{\end{snugshade}}
\newcommand{\AlertTok}[1]{\textcolor[rgb]{0.94,0.16,0.16}{#1}}
\newcommand{\AnnotationTok}[1]{\textcolor[rgb]{0.56,0.35,0.01}{\textbf{\textit{#1}}}}
\newcommand{\AttributeTok}[1]{\textcolor[rgb]{0.13,0.29,0.53}{#1}}
\newcommand{\BaseNTok}[1]{\textcolor[rgb]{0.00,0.00,0.81}{#1}}
\newcommand{\BuiltInTok}[1]{#1}
\newcommand{\CharTok}[1]{\textcolor[rgb]{0.31,0.60,0.02}{#1}}
\newcommand{\CommentTok}[1]{\textcolor[rgb]{0.56,0.35,0.01}{\textit{#1}}}
\newcommand{\CommentVarTok}[1]{\textcolor[rgb]{0.56,0.35,0.01}{\textbf{\textit{#1}}}}
\newcommand{\ConstantTok}[1]{\textcolor[rgb]{0.56,0.35,0.01}{#1}}
\newcommand{\ControlFlowTok}[1]{\textcolor[rgb]{0.13,0.29,0.53}{\textbf{#1}}}
\newcommand{\DataTypeTok}[1]{\textcolor[rgb]{0.13,0.29,0.53}{#1}}
\newcommand{\DecValTok}[1]{\textcolor[rgb]{0.00,0.00,0.81}{#1}}
\newcommand{\DocumentationTok}[1]{\textcolor[rgb]{0.56,0.35,0.01}{\textbf{\textit{#1}}}}
\newcommand{\ErrorTok}[1]{\textcolor[rgb]{0.64,0.00,0.00}{\textbf{#1}}}
\newcommand{\ExtensionTok}[1]{#1}
\newcommand{\FloatTok}[1]{\textcolor[rgb]{0.00,0.00,0.81}{#1}}
\newcommand{\FunctionTok}[1]{\textcolor[rgb]{0.13,0.29,0.53}{\textbf{#1}}}
\newcommand{\ImportTok}[1]{#1}
\newcommand{\InformationTok}[1]{\textcolor[rgb]{0.56,0.35,0.01}{\textbf{\textit{#1}}}}
\newcommand{\KeywordTok}[1]{\textcolor[rgb]{0.13,0.29,0.53}{\textbf{#1}}}
\newcommand{\NormalTok}[1]{#1}
\newcommand{\OperatorTok}[1]{\textcolor[rgb]{0.81,0.36,0.00}{\textbf{#1}}}
\newcommand{\OtherTok}[1]{\textcolor[rgb]{0.56,0.35,0.01}{#1}}
\newcommand{\PreprocessorTok}[1]{\textcolor[rgb]{0.56,0.35,0.01}{\textit{#1}}}
\newcommand{\RegionMarkerTok}[1]{#1}
\newcommand{\SpecialCharTok}[1]{\textcolor[rgb]{0.81,0.36,0.00}{\textbf{#1}}}
\newcommand{\SpecialStringTok}[1]{\textcolor[rgb]{0.31,0.60,0.02}{#1}}
\newcommand{\StringTok}[1]{\textcolor[rgb]{0.31,0.60,0.02}{#1}}
\newcommand{\VariableTok}[1]{\textcolor[rgb]{0.00,0.00,0.00}{#1}}
\newcommand{\VerbatimStringTok}[1]{\textcolor[rgb]{0.31,0.60,0.02}{#1}}
\newcommand{\WarningTok}[1]{\textcolor[rgb]{0.56,0.35,0.01}{\textbf{\textit{#1}}}}
\usepackage{longtable,booktabs,array}
\usepackage{calc} % for calculating minipage widths
% Correct order of tables after \paragraph or \subparagraph
\usepackage{etoolbox}
\makeatletter
\patchcmd\longtable{\par}{\if@noskipsec\mbox{}\fi\par}{}{}
\makeatother
% Allow footnotes in longtable head/foot
\IfFileExists{footnotehyper.sty}{\usepackage{footnotehyper}}{\usepackage{footnote}}
\makesavenoteenv{longtable}
\usepackage{graphicx}
\makeatletter
\def\maxwidth{\ifdim\Gin@nat@width>\linewidth\linewidth\else\Gin@nat@width\fi}
\def\maxheight{\ifdim\Gin@nat@height>\textheight\textheight\else\Gin@nat@height\fi}
\makeatother
% Scale images if necessary, so that they will not overflow the page
% margins by default, and it is still possible to overwrite the defaults
% using explicit options in \includegraphics[width, height, ...]{}
\setkeys{Gin}{width=\maxwidth,height=\maxheight,keepaspectratio}
% Set default figure placement to htbp
\makeatletter
\def\fps@figure{htbp}
\makeatother
\setlength{\emergencystretch}{3em} % prevent overfull lines
\providecommand{\tightlist}{%
  \setlength{\itemsep}{0pt}\setlength{\parskip}{0pt}}
\setcounter{secnumdepth}{-\maxdimen} % remove section numbering
\ifLuaTeX
  \usepackage{selnolig}  % disable illegal ligatures
\fi
\usepackage{bookmark}
\IfFileExists{xurl.sty}{\usepackage{xurl}}{} % add URL line breaks if available
\urlstyle{same}
\hypersetup{
  pdftitle={Exercise 3 solution},
  pdfauthor={Tristan, Köning, Ali Movasati, Isabelle Cretton},
  hidelinks,
  pdfcreator={LaTeX via pandoc}}

\title{Exercise 3 solution}
\author{Tristan, Köning, Ali Movasati, Isabelle Cretton}
\date{Oct.~1st, 2024}

\begin{document}
\maketitle

\begin{Shaded}
\begin{Highlighting}[]
\CommentTok{\# Set global code chunk options}
\NormalTok{knitr}\SpecialCharTok{::}\NormalTok{opts\_chunk}\SpecialCharTok{$}\FunctionTok{set}\NormalTok{(}\AttributeTok{warning =} \ConstantTok{FALSE}\NormalTok{)}
\end{Highlighting}
\end{Shaded}

\begin{Shaded}
\begin{Highlighting}[]
\FunctionTok{library}\NormalTok{(cluster)}
\FunctionTok{library}\NormalTok{(stats)}
\FunctionTok{library}\NormalTok{(party)}
\end{Highlighting}
\end{Shaded}

\begin{verbatim}
## Loading required package: grid
\end{verbatim}

\begin{verbatim}
## Loading required package: mvtnorm
\end{verbatim}

\begin{verbatim}
## Loading required package: modeltools
\end{verbatim}

\begin{verbatim}
## Loading required package: stats4
\end{verbatim}

\begin{verbatim}
## Loading required package: strucchange
\end{verbatim}

\begin{verbatim}
## Loading required package: zoo
\end{verbatim}

\begin{verbatim}
## 
## Attaching package: 'zoo'
\end{verbatim}

\begin{verbatim}
## The following objects are masked from 'package:base':
## 
##     as.Date, as.Date.numeric
\end{verbatim}

\begin{verbatim}
## Loading required package: sandwich
\end{verbatim}

\begin{Shaded}
\begin{Highlighting}[]
\FunctionTok{library}\NormalTok{(dplyr)}
\end{Highlighting}
\end{Shaded}

\begin{verbatim}
## 
## Attaching package: 'dplyr'
\end{verbatim}

\begin{verbatim}
## The following object is masked from 'package:party':
## 
##     where
\end{verbatim}

\begin{verbatim}
## The following objects are masked from 'package:stats':
## 
##     filter, lag
\end{verbatim}

\begin{verbatim}
## The following objects are masked from 'package:base':
## 
##     intersect, setdiff, setequal, union
\end{verbatim}

\begin{Shaded}
\begin{Highlighting}[]
\FunctionTok{library}\NormalTok{(ggplot2)}
\CommentTok{\#install.packages("skimr")}
\FunctionTok{library}\NormalTok{(skimr)}
\FunctionTok{library}\NormalTok{(tidyverse)}
\end{Highlighting}
\end{Shaded}

\begin{verbatim}
## -- Attaching core tidyverse packages ------------------------ tidyverse 2.0.0 --
## v forcats   1.0.0     v stringr   1.5.1
## v lubridate 1.9.3     v tibble    3.2.1
## v purrr     1.0.2     v tidyr     1.3.1
## v readr     2.1.5
\end{verbatim}

\begin{verbatim}
## -- Conflicts ------------------------------------------ tidyverse_conflicts() --
## x stringr::boundary() masks strucchange::boundary()
## x dplyr::filter()     masks stats::filter()
## x dplyr::lag()        masks stats::lag()
## x dplyr::where()      masks party::where()
## i Use the conflicted package (<http://conflicted.r-lib.org/>) to force all conflicts to become errors
\end{verbatim}

\begin{Shaded}
\begin{Highlighting}[]
\FunctionTok{library}\NormalTok{(ggrepel)}
\CommentTok{\#library for LDA analysis}
\FunctionTok{library}\NormalTok{(MASS)}
\end{Highlighting}
\end{Shaded}

\begin{verbatim}
## 
## Attaching package: 'MASS'
## 
## The following object is masked from 'package:dplyr':
## 
##     select
\end{verbatim}

\begin{Shaded}
\begin{Highlighting}[]
\CommentTok{\#libraries for random forests and classification trees}
\FunctionTok{library}\NormalTok{(rpart)        }\CommentTok{\# For classification trees}
\FunctionTok{library}\NormalTok{(rpart.plot)}
\FunctionTok{library}\NormalTok{(randomForest)       }\CommentTok{\# For data splitting and accuracy evaluation}
\end{Highlighting}
\end{Shaded}

\begin{verbatim}
## randomForest 4.7-1.2
## Type rfNews() to see new features/changes/bug fixes.
## 
## Attaching package: 'randomForest'
## 
## The following object is masked from 'package:ggplot2':
## 
##     margin
## 
## The following object is masked from 'package:dplyr':
## 
##     combine
\end{verbatim}

\begin{Shaded}
\begin{Highlighting}[]
\FunctionTok{library}\NormalTok{(caret)        }\CommentTok{\# For data splitting and accuracy evaluation}
\end{Highlighting}
\end{Shaded}

\begin{verbatim}
## Loading required package: lattice
## 
## Attaching package: 'caret'
## 
## The following object is masked from 'package:purrr':
## 
##     lift
\end{verbatim}

\section{Problem 1}\label{problem-1}

\subsection{Load the txt file}\label{load-the-txt-file}

\begin{Shaded}
\begin{Highlighting}[]
\CommentTok{\# Load the data}
\NormalTok{ec\_data }\OtherTok{\textless{}{-}} \FunctionTok{read.csv}\NormalTok{(}\StringTok{"data/ec.txt"}\NormalTok{, }\AttributeTok{header =} \ConstantTok{TRUE}\NormalTok{)}
\end{Highlighting}
\end{Shaded}

\subsection{1.A}\label{a}

\subsubsection{explore and visualize the data w/ plots and
tables}\label{explore-and-visualize-the-data-w-plots-and-tables}

\begin{Shaded}
\begin{Highlighting}[]
\CommentTok{\# Summary of the data}
\FunctionTok{skim}\NormalTok{(ec\_data)}
\end{Highlighting}
\end{Shaded}

\begin{longtable}[]{@{}ll@{}}
\caption{Data summary}\tabularnewline
\toprule\noalign{}
\endfirsthead
\endhead
\bottomrule\noalign{}
\endlastfoot
Name & ec\_data \\
Number of rows & 48 \\
Number of columns & 5 \\
\_\_\_\_\_\_\_\_\_\_\_\_\_\_\_\_\_\_\_\_\_\_\_ & \\
Column type frequency: & \\
character & 2 \\
numeric & 3 \\
\_\_\_\_\_\_\_\_\_\_\_\_\_\_\_\_\_\_\_\_\_\_\_\_ & \\
Group variables & None \\
\end{longtable}

\textbf{Variable type: character}

\begin{longtable}[]{@{}
  >{\raggedright\arraybackslash}p{(\columnwidth - 14\tabcolsep) * \real{0.1944}}
  >{\raggedleft\arraybackslash}p{(\columnwidth - 14\tabcolsep) * \real{0.1389}}
  >{\raggedleft\arraybackslash}p{(\columnwidth - 14\tabcolsep) * \real{0.1944}}
  >{\raggedleft\arraybackslash}p{(\columnwidth - 14\tabcolsep) * \real{0.0556}}
  >{\raggedleft\arraybackslash}p{(\columnwidth - 14\tabcolsep) * \real{0.0556}}
  >{\raggedleft\arraybackslash}p{(\columnwidth - 14\tabcolsep) * \real{0.0833}}
  >{\raggedleft\arraybackslash}p{(\columnwidth - 14\tabcolsep) * \real{0.1250}}
  >{\raggedleft\arraybackslash}p{(\columnwidth - 14\tabcolsep) * \real{0.1528}}@{}}
\toprule\noalign{}
\begin{minipage}[b]{\linewidth}\raggedright
skim\_variable
\end{minipage} & \begin{minipage}[b]{\linewidth}\raggedleft
n\_missing
\end{minipage} & \begin{minipage}[b]{\linewidth}\raggedleft
complete\_rate
\end{minipage} & \begin{minipage}[b]{\linewidth}\raggedleft
min
\end{minipage} & \begin{minipage}[b]{\linewidth}\raggedleft
max
\end{minipage} & \begin{minipage}[b]{\linewidth}\raggedleft
empty
\end{minipage} & \begin{minipage}[b]{\linewidth}\raggedleft
n\_unique
\end{minipage} & \begin{minipage}[b]{\linewidth}\raggedleft
whitespace
\end{minipage} \\
\midrule\noalign{}
\endhead
\bottomrule\noalign{}
\endlastfoot
City & 0 & 1.00 & 4 & 14 & 0 & 48 & 0 \\
SalaryCat & 2 & 0.96 & 10 & 11 & 0 & 3 & 0 \\
\end{longtable}

\textbf{Variable type: numeric}

\begin{longtable}[]{@{}
  >{\raggedright\arraybackslash}p{(\columnwidth - 20\tabcolsep) * \real{0.1443}}
  >{\raggedleft\arraybackslash}p{(\columnwidth - 20\tabcolsep) * \real{0.1031}}
  >{\raggedleft\arraybackslash}p{(\columnwidth - 20\tabcolsep) * \real{0.1443}}
  >{\raggedleft\arraybackslash}p{(\columnwidth - 20\tabcolsep) * \real{0.0825}}
  >{\raggedleft\arraybackslash}p{(\columnwidth - 20\tabcolsep) * \real{0.0722}}
  >{\raggedleft\arraybackslash}p{(\columnwidth - 20\tabcolsep) * \real{0.0722}}
  >{\raggedleft\arraybackslash}p{(\columnwidth - 20\tabcolsep) * \real{0.0825}}
  >{\raggedleft\arraybackslash}p{(\columnwidth - 20\tabcolsep) * \real{0.0825}}
  >{\raggedleft\arraybackslash}p{(\columnwidth - 20\tabcolsep) * \real{0.0825}}
  >{\raggedleft\arraybackslash}p{(\columnwidth - 20\tabcolsep) * \real{0.0722}}
  >{\raggedright\arraybackslash}p{(\columnwidth - 20\tabcolsep) * \real{0.0619}}@{}}
\toprule\noalign{}
\begin{minipage}[b]{\linewidth}\raggedright
skim\_variable
\end{minipage} & \begin{minipage}[b]{\linewidth}\raggedleft
n\_missing
\end{minipage} & \begin{minipage}[b]{\linewidth}\raggedleft
complete\_rate
\end{minipage} & \begin{minipage}[b]{\linewidth}\raggedleft
mean
\end{minipage} & \begin{minipage}[b]{\linewidth}\raggedleft
sd
\end{minipage} & \begin{minipage}[b]{\linewidth}\raggedleft
p0
\end{minipage} & \begin{minipage}[b]{\linewidth}\raggedleft
p25
\end{minipage} & \begin{minipage}[b]{\linewidth}\raggedleft
p50
\end{minipage} & \begin{minipage}[b]{\linewidth}\raggedleft
p75
\end{minipage} & \begin{minipage}[b]{\linewidth}\raggedleft
p100
\end{minipage} & \begin{minipage}[b]{\linewidth}\raggedright
hist
\end{minipage} \\
\midrule\noalign{}
\endhead
\bottomrule\noalign{}
\endlastfoot
Work & 2 & 0.96 & 1879.91 & 174.34 & 1583.0 & 1745.25 & 1849.00 &
1976.25 & 2375.0 & ▅▇▅▂▁ \\
Price & 0 & 1.00 & 68.86 & 21.78 & 30.3 & 49.65 & 70.50 & 81.70 & 115.5
& ▅▆▇▅▂ \\
Salary & 2 & 0.96 & 39.55 & 24.76 & 2.7 & 14.38 & 43.65 & 59.70 & 100.0
& ▇▅▇▅▁ \\
\end{longtable}

\begin{Shaded}
\begin{Highlighting}[]
\NormalTok{ec\_data\_clean }\OtherTok{\textless{}{-}}\NormalTok{ ec\_data }\SpecialCharTok{\%\textgreater{}\%} \FunctionTok{na.omit}\NormalTok{()}
\end{Highlighting}
\end{Shaded}

\begin{Shaded}
\begin{Highlighting}[]
\CommentTok{\# Scatterplot of the data}
\FunctionTok{ggplot}\NormalTok{(ec\_data, }\FunctionTok{aes}\NormalTok{(}\AttributeTok{x =}\NormalTok{ Work, }\AttributeTok{y =}\NormalTok{ Salary, }\AttributeTok{color =}\NormalTok{ Price)) }\SpecialCharTok{+}
  \FunctionTok{geom\_point}\NormalTok{(}\AttributeTok{size =} \DecValTok{3}\NormalTok{, }\AttributeTok{alpha =} \FloatTok{0.7}\NormalTok{) }\SpecialCharTok{+}
  \FunctionTok{geom\_smooth}\NormalTok{(}\AttributeTok{method =} \StringTok{"lm"}\NormalTok{, }\AttributeTok{se =} \ConstantTok{FALSE}\NormalTok{, }\AttributeTok{color =} \StringTok{"red"}\NormalTok{) }\SpecialCharTok{+} 
  \FunctionTok{scale\_color\_gradient}\NormalTok{(}\AttributeTok{low =} \StringTok{"blue"}\NormalTok{, }\AttributeTok{high =} \StringTok{"red"}\NormalTok{) }\SpecialCharTok{+}
  \FunctionTok{labs}\NormalTok{(}\AttributeTok{title =} \StringTok{"Scatterplot of Work vs Salary"}\NormalTok{,}
       \AttributeTok{x =} \StringTok{"Work (Weighted Average of Working Hours)"}\NormalTok{,}
       \AttributeTok{y =} \StringTok{"Salary (Index)"}\NormalTok{,}
       \AttributeTok{color =} \StringTok{"Price Index"}\NormalTok{) }\SpecialCharTok{+}
  \FunctionTok{theme\_minimal}\NormalTok{()}
\end{Highlighting}
\end{Shaded}

\begin{verbatim}
## `geom_smooth()` using formula = 'y ~ x'
\end{verbatim}

\includegraphics{day4_Final_files/figure-latex/unnamed-chunk-3-1.pdf}
This plot shows a negative correlation between Work (average working
hours) and Salary (index of earnings). Cities with higher working hours
tend to have lower salaries. There seems to be a consistent trend where
cities that require more work hours offer lower salaries, with Zurich
(salary = 100) standing out as an exception with a lower work average
and higher salary. Prices generally correlate positively with salaries,
though not perfectly.

\begin{Shaded}
\begin{Highlighting}[]
\CommentTok{\# Boxplot of the data}
\NormalTok{boxplot\_data }\OtherTok{\textless{}{-}}\NormalTok{ ec\_data\_clean }\SpecialCharTok{\%\textgreater{}\%} 
  \FunctionTok{pivot\_longer}\NormalTok{(}\AttributeTok{cols =} \FunctionTok{c}\NormalTok{(Price, Salary), }
               \AttributeTok{names\_to =} \StringTok{"Variable"}\NormalTok{, }
               \AttributeTok{values\_to =} \StringTok{"Index"}\NormalTok{)}

\CommentTok{\# Plot the boxplot}
\FunctionTok{ggplot}\NormalTok{(boxplot\_data, }\FunctionTok{aes}\NormalTok{(}\AttributeTok{x =}\NormalTok{ Variable, }\AttributeTok{y =}\NormalTok{ Index, }\AttributeTok{fill =}\NormalTok{ Variable)) }\SpecialCharTok{+}
  \FunctionTok{geom\_boxplot}\NormalTok{() }\SpecialCharTok{+}
  \FunctionTok{scale\_fill\_manual}\NormalTok{(}\AttributeTok{values =} \FunctionTok{c}\NormalTok{(}\StringTok{"Price"} \OtherTok{=} \StringTok{"\#FF9999"}\NormalTok{, }\StringTok{"Salary"} \OtherTok{=} \StringTok{"\#9999FF"}\NormalTok{)) }\SpecialCharTok{+}
  \FunctionTok{labs}\NormalTok{(}\AttributeTok{title =} \StringTok{"Boxplot of Price and Salary Indices"}\NormalTok{) }\SpecialCharTok{+}
  \FunctionTok{theme\_minimal}\NormalTok{()}
\end{Highlighting}
\end{Shaded}

\includegraphics{day4_Final_files/figure-latex/unnamed-chunk-4-1.pdf}
The boxplot compares the distribution of Price and Salary. Price has a
narrower spread than Salary, with a higher median value
(\textasciitilde70) compared to the median salary (\textasciitilde40).
The greater variability in Salary indicates that wage levels are more
diverse across cities than price levels. Many cities face low salary
levels despite high prices, emphasizing the cost of living challenges in
certain cities.

\begin{Shaded}
\begin{Highlighting}[]
\CommentTok{\# Density plot of the data}
\FunctionTok{ggplot}\NormalTok{(ec\_data\_clean, }\FunctionTok{aes}\NormalTok{(}\AttributeTok{x =}\NormalTok{ Work, }\AttributeTok{fill =}\NormalTok{ SalaryCat)) }\SpecialCharTok{+}
  \FunctionTok{geom\_density}\NormalTok{(}\AttributeTok{alpha =} \FloatTok{0.5}\NormalTok{) }\SpecialCharTok{+}
  \FunctionTok{labs}\NormalTok{(}\AttributeTok{title =} \StringTok{"Density Plot of Work Hours by Salary Category"}\NormalTok{,}
       \AttributeTok{x =} \StringTok{"Work (Weighted Average of Working Hours)"}\NormalTok{,}
       \AttributeTok{fill =} \StringTok{"Salary Category"}\NormalTok{) }\SpecialCharTok{+}
  \FunctionTok{theme\_minimal}\NormalTok{()}
\end{Highlighting}
\end{Shaded}

\includegraphics{day4_Final_files/figure-latex/unnamed-chunk-5-1.pdf}
This plot shows the distribution of Work (working hours) broken down by
SalaryCat (salary categories). The highest salary category (67.6, 100{]}
is associated with cities that have a narrow peak in work hours, with a
median around 1900-2000 hours. Conversely, lower salary categories (2.6,
35.1{]} and (35.1, 67.6{]} are spread across a wider range of work
hours, with some cities having very high work hours. Cities with higher
salaries tend to have more standardized working hours, while cities with
lower salaries show a broader spread in work hours, including cities
with extremely high work burdens.

\begin{Shaded}
\begin{Highlighting}[]
\FunctionTok{colnames}\NormalTok{(ec\_data\_clean)}
\end{Highlighting}
\end{Shaded}

\begin{verbatim}
## [1] "City"      "Work"      "Price"     "Salary"    "SalaryCat"
\end{verbatim}

\subsection{1.B}\label{b}

\begin{Shaded}
\begin{Highlighting}[]
\CommentTok{\# (remove non{-}numeric columns and scale)}
\NormalTok{cluster\_data }\OtherTok{\textless{}{-}}\NormalTok{ ec\_data\_clean }\SpecialCharTok{\%\textgreater{}\%}
\NormalTok{  dplyr}\SpecialCharTok{::}\FunctionTok{select}\NormalTok{(Work, Price, Salary) }\SpecialCharTok{\%\textgreater{}\%}
  \FunctionTok{scale}\NormalTok{()  }\CommentTok{\# standardize the data to mean 0 and variance 1}

\FunctionTok{set.seed}\NormalTok{(}\DecValTok{111}\NormalTok{)  }\CommentTok{\# Set seed for reproducibility}
\NormalTok{wss }\OtherTok{\textless{}{-}} \FunctionTok{sapply}\NormalTok{(}\DecValTok{1}\SpecialCharTok{:}\DecValTok{10}\NormalTok{, }\ControlFlowTok{function}\NormalTok{(k) \{ }\CommentTok{\#within{-}cluster sum of squares}
  \FunctionTok{kmeans}\NormalTok{(cluster\_data, }\AttributeTok{centers =}\NormalTok{ k, }\AttributeTok{nstart =} \DecValTok{20}\NormalTok{)}\SpecialCharTok{$}\NormalTok{tot.withinss}
\NormalTok{\})}

\CommentTok{\# Plot the elbow method}
\FunctionTok{ggplot}\NormalTok{(}\FunctionTok{data.frame}\NormalTok{(}\AttributeTok{k =} \DecValTok{1}\SpecialCharTok{:}\DecValTok{10}\NormalTok{, }\AttributeTok{WSS =}\NormalTok{ wss), }\FunctionTok{aes}\NormalTok{(}\AttributeTok{x =}\NormalTok{ k, }\AttributeTok{y =}\NormalTok{ WSS)) }\SpecialCharTok{+}
  \FunctionTok{geom\_line}\NormalTok{(}\AttributeTok{color =} \StringTok{"blue"}\NormalTok{) }\SpecialCharTok{+}
  \FunctionTok{geom\_point}\NormalTok{(}\AttributeTok{color =} \StringTok{"red"}\NormalTok{) }\SpecialCharTok{+}
  \FunctionTok{labs}\NormalTok{(}\AttributeTok{title =} \StringTok{"Elbow Method for K{-}means Clustering"}\NormalTok{,}
       \AttributeTok{x =} \StringTok{"Number of Clusters (K)"}\NormalTok{,}
       \AttributeTok{y =} \StringTok{"Total Within{-}cluster Sum of Squares (WSS)"}\NormalTok{) }\SpecialCharTok{+}
  \FunctionTok{theme\_minimal}\NormalTok{()}
\end{Highlighting}
\end{Shaded}

\includegraphics{day4_Final_files/figure-latex/unnamed-chunk-7-1.pdf}

\begin{Shaded}
\begin{Highlighting}[]
\FunctionTok{set.seed}\NormalTok{(}\DecValTok{111}\NormalTok{)}
\NormalTok{kmeans\_result }\OtherTok{\textless{}{-}} \FunctionTok{kmeans}\NormalTok{(cluster\_data, }\AttributeTok{centers =} \DecValTok{3}\NormalTok{, }\AttributeTok{nstart =} \DecValTok{20}\NormalTok{)}

\CommentTok{\# Add cluster assignment to the data}
\NormalTok{ec\_data\_clean}\SpecialCharTok{$}\NormalTok{cluster }\OtherTok{\textless{}{-}} \FunctionTok{as.factor}\NormalTok{(kmeans\_result}\SpecialCharTok{$}\NormalTok{cluster)}

\CommentTok{\# Visualize clusters}
\FunctionTok{ggplot}\NormalTok{(ec\_data\_clean, }\FunctionTok{aes}\NormalTok{(}\AttributeTok{x =}\NormalTok{ Work, }\AttributeTok{y =}\NormalTok{ Salary, }\AttributeTok{color =}\NormalTok{ cluster)) }\SpecialCharTok{+}
  \FunctionTok{geom\_point}\NormalTok{(}\AttributeTok{size =} \DecValTok{3}\NormalTok{) }\SpecialCharTok{+}
  \FunctionTok{geom\_text}\NormalTok{(}\FunctionTok{aes}\NormalTok{(}\AttributeTok{label =}\NormalTok{ City), }\AttributeTok{hjust =} \FloatTok{1.2}\NormalTok{, }\AttributeTok{vjust =} \SpecialCharTok{{-}}\FloatTok{0.5}\NormalTok{, }\AttributeTok{size =} \DecValTok{3}\NormalTok{) }\SpecialCharTok{+}
  \FunctionTok{labs}\NormalTok{(}\AttributeTok{title =} \StringTok{"Cluster Analysis of Cities Based on Work, Salary, and Price"}\NormalTok{,}
       \AttributeTok{x =} \StringTok{"Work (Weighted Average of Working Hours)"}\NormalTok{,}
       \AttributeTok{y =} \StringTok{"Salary (Index)"}\NormalTok{) }\SpecialCharTok{+}
  \FunctionTok{theme\_minimal}\NormalTok{()}
\end{Highlighting}
\end{Shaded}

\includegraphics{day4_Final_files/figure-latex/unnamed-chunk-7-2.pdf} -
Cluster 1 (Red): Cities in this cluster (e.g., Hong Kong, Manila,
Bogota) are characterized by high work hours and low salaries. They
might represent emerging markets where the cost of living and salaries
are lower, but work demands are higher. - Cluster 2 (Blue): Cities like
Mexico City, Seoul, and Athens are similar in that they have moderate
working hours and moderate to low salaries. These cities often show a
developing or transitional economic status. - Cluster 3 (Green): Cities
like Zurich, New York, and Tokyo are similar in offering high salaries
and lower working hours, suggesting that they are developed economies
with better work-life balance and higher living standards.

\subsection{1.C}\label{c}

\begin{Shaded}
\begin{Highlighting}[]
\NormalTok{pca }\OtherTok{\textless{}{-}} \FunctionTok{prcomp}\NormalTok{(cluster\_data, }\AttributeTok{scale =} \ConstantTok{TRUE}\NormalTok{)}

\NormalTok{loadings }\OtherTok{\textless{}{-}}\NormalTok{ pca}\SpecialCharTok{$}\NormalTok{rotation}
\NormalTok{loadings\_data }\OtherTok{\textless{}{-}} \FunctionTok{data.frame}\NormalTok{(}\AttributeTok{Variable =} \FunctionTok{rownames}\NormalTok{(loadings), }\AttributeTok{PC1 =}\NormalTok{ loadings[, }\DecValTok{1}\NormalTok{], }\AttributeTok{PC2 =}\NormalTok{ loadings[, }\DecValTok{2}\NormalTok{])}

\CommentTok{\# perform Dimensionality reduction}
\NormalTok{scores }\OtherTok{\textless{}{-}}\NormalTok{ pca}\SpecialCharTok{$}\NormalTok{x}
\NormalTok{pca\_data }\OtherTok{\textless{}{-}} \FunctionTok{data.frame}\NormalTok{(}\AttributeTok{City =}\NormalTok{ ec\_data\_clean}\SpecialCharTok{$}\NormalTok{City, }\AttributeTok{PC1 =}\NormalTok{ scores[, }\DecValTok{1}\NormalTok{], }\AttributeTok{PC2 =}\NormalTok{ scores[, }\DecValTok{2}\NormalTok{])}

\CommentTok{\# plot the PCA biplot}
\FunctionTok{ggplot}\NormalTok{() }\SpecialCharTok{+} 
  \FunctionTok{geom\_point}\NormalTok{(}\AttributeTok{data =}\NormalTok{ pca\_data, }\FunctionTok{aes}\NormalTok{(}\AttributeTok{x =}\NormalTok{ PC1, }\AttributeTok{y =}\NormalTok{ PC2, }\AttributeTok{color =}\NormalTok{ ec\_data\_clean}\SpecialCharTok{$}\NormalTok{cluster)) }\SpecialCharTok{+} 
  \FunctionTok{geom\_text\_repel}\NormalTok{(}\AttributeTok{data =}\NormalTok{ pca\_data, }\FunctionTok{aes}\NormalTok{(}\AttributeTok{x =}\NormalTok{ PC1, }\AttributeTok{y =}\NormalTok{ PC2, }\AttributeTok{label =}\NormalTok{ City)) }\SpecialCharTok{+}
  \FunctionTok{geom\_segment}\NormalTok{(}\AttributeTok{data =}\NormalTok{ loadings\_data, }\FunctionTok{aes}\NormalTok{(}\AttributeTok{x =} \DecValTok{0}\NormalTok{, }\AttributeTok{y =} \DecValTok{0}\NormalTok{, }\AttributeTok{xend =}\NormalTok{ PC1, }\AttributeTok{yend =}\NormalTok{ PC2),}
               \AttributeTok{arrow =} \FunctionTok{arrow}\NormalTok{(}\AttributeTok{length =} \FunctionTok{unit}\NormalTok{(}\FloatTok{0.3}\NormalTok{, }\StringTok{"cm"}\NormalTok{)), }\AttributeTok{color =} \StringTok{"red"}\NormalTok{) }\SpecialCharTok{+}
  \FunctionTok{geom\_text\_repel}\NormalTok{(}\AttributeTok{data =}\NormalTok{ loadings\_data, }\FunctionTok{aes}\NormalTok{(}\AttributeTok{x =}\NormalTok{ PC1, }\AttributeTok{y =}\NormalTok{ PC2, }\AttributeTok{label =}\NormalTok{ Variable), }\AttributeTok{color =} \StringTok{"red"}\NormalTok{) }\SpecialCharTok{+}
  \FunctionTok{labs}\NormalTok{(}\AttributeTok{title =} \StringTok{"PCA Biplot of Cities Based on Work, Salary, and Price"}\NormalTok{,}
       \AttributeTok{x =} \StringTok{"PC1"}\NormalTok{,}
       \AttributeTok{y =} \StringTok{"PC2"}\NormalTok{) }\SpecialCharTok{+}
  \FunctionTok{theme\_minimal}\NormalTok{()}
\end{Highlighting}
\end{Shaded}

\includegraphics{day4_Final_files/figure-latex/unnamed-chunk-8-1.pdf}
The PCA biplot reveals how cities differ in terms of Work, Salary, and
Price by projecting them onto two principal components (PC1 and PC2).
PC1 primarily captures the variation in working hours, with cities like
Hong Kong and Manila positioned on the right, indicating high working
hours, while cities like Zurich and Geneva are on the left, showing
fewer working hours. PC2 captures the variation in salary and price,
with cities like Lisbon and Rio de Janeiro higher on the plot,
indicating higher salary and price levels, while Bombay and Seoul show
lower values. The biplot visually confirms the clustering pattern, where
cities with similar work-life and economic profiles are grouped
together, such as high-salary, low-work cities in Cluster 3 (green) and
high-work, low-salary cities in Cluster 1 (red).

\subsection{1.D (LDA \& QDA)}\label{d-lda-qda}

\begin{Shaded}
\begin{Highlighting}[]
\NormalTok{data }\OtherTok{\textless{}{-}}\NormalTok{ ec\_data\_clean }\SpecialCharTok{\%\textgreater{}\%}
\NormalTok{  dplyr}\SpecialCharTok{::}\FunctionTok{select}\NormalTok{(Work, Price, SalaryCat) }\SpecialCharTok{\%\textgreater{}\%}
  \FunctionTok{na.omit}\NormalTok{()  }\CommentTok{\# Remove rows with missing data}

\CommentTok{\# Convert SalaryCat to a factor}
\NormalTok{data}\SpecialCharTok{$}\NormalTok{SalaryCat }\OtherTok{\textless{}{-}} \FunctionTok{as.factor}\NormalTok{(data}\SpecialCharTok{$}\NormalTok{SalaryCat)}

\CommentTok{\# LDA model}
\NormalTok{lda\_model }\OtherTok{\textless{}{-}} \FunctionTok{lda}\NormalTok{(SalaryCat }\SpecialCharTok{\textasciitilde{}}\NormalTok{ Work }\SpecialCharTok{+}\NormalTok{ Price, }\AttributeTok{data =}\NormalTok{ data)}

\CommentTok{\# Create a grid of values for Work and Price to plot decision boundaries}
\NormalTok{x\_min }\OtherTok{\textless{}{-}} \FunctionTok{min}\NormalTok{(data}\SpecialCharTok{$}\NormalTok{Work) }\SpecialCharTok{{-}} \DecValTok{1}
\NormalTok{x\_max }\OtherTok{\textless{}{-}} \FunctionTok{max}\NormalTok{(data}\SpecialCharTok{$}\NormalTok{Work) }\SpecialCharTok{+} \DecValTok{1}
\NormalTok{y\_min }\OtherTok{\textless{}{-}} \FunctionTok{min}\NormalTok{(data}\SpecialCharTok{$}\NormalTok{Price) }\SpecialCharTok{{-}} \DecValTok{1}
\NormalTok{y\_max }\OtherTok{\textless{}{-}} \FunctionTok{max}\NormalTok{(data}\SpecialCharTok{$}\NormalTok{Price) }\SpecialCharTok{+} \DecValTok{1}

\NormalTok{grid }\OtherTok{\textless{}{-}} \FunctionTok{expand.grid}\NormalTok{(}\AttributeTok{Work =} \FunctionTok{seq}\NormalTok{(x\_min, x\_max, }\AttributeTok{length.out =} \DecValTok{100}\NormalTok{),}
                    \AttributeTok{Price =} \FunctionTok{seq}\NormalTok{(y\_min, y\_max, }\AttributeTok{length.out =} \DecValTok{100}\NormalTok{))}

\CommentTok{\# Predict class probabilities and categories for LDA and QDA}
\NormalTok{lda\_pred }\OtherTok{\textless{}{-}} \FunctionTok{predict}\NormalTok{(lda\_model, grid)}
\NormalTok{grid}\SpecialCharTok{$}\NormalTok{lda\_class }\OtherTok{\textless{}{-}}\NormalTok{ lda\_pred}\SpecialCharTok{$}\NormalTok{class}

\CommentTok{\# Plot LDA decision boundaries}
\NormalTok{lda\_plot }\OtherTok{\textless{}{-}} \FunctionTok{ggplot}\NormalTok{() }\SpecialCharTok{+} 
  \FunctionTok{geom\_point}\NormalTok{(}\AttributeTok{data =}\NormalTok{ grid, }\FunctionTok{aes}\NormalTok{(}\AttributeTok{x =}\NormalTok{ Work, }\AttributeTok{y =}\NormalTok{ Price, }\AttributeTok{color =}\NormalTok{ lda\_class), }\AttributeTok{alpha =} \FloatTok{0.3}\NormalTok{) }\SpecialCharTok{+} 
  \FunctionTok{geom\_point}\NormalTok{(}\AttributeTok{data =}\NormalTok{ data, }\FunctionTok{aes}\NormalTok{(}\AttributeTok{x =}\NormalTok{ Work, }\AttributeTok{y =}\NormalTok{ Price, }\AttributeTok{color =}\NormalTok{ SalaryCat)) }\SpecialCharTok{+}
  \FunctionTok{labs}\NormalTok{(}\AttributeTok{title =} \StringTok{"LDA: Work vs Price Decision Boundary"}\NormalTok{,}
       \AttributeTok{x =} \StringTok{"Work (Weighted Average of Working Hours)"}\NormalTok{,}
       \AttributeTok{y =} \StringTok{"Price (Index)"}\NormalTok{) }\SpecialCharTok{+}
  \FunctionTok{theme\_minimal}\NormalTok{()}

\NormalTok{lda\_plot}
\end{Highlighting}
\end{Shaded}

\includegraphics{day4_Final_files/figure-latex/unnamed-chunk-9-1.pdf}

\begin{Shaded}
\begin{Highlighting}[]
\CommentTok{\#qda}
\NormalTok{qda\_model }\OtherTok{\textless{}{-}} \FunctionTok{qda}\NormalTok{(SalaryCat }\SpecialCharTok{\textasciitilde{}}\NormalTok{ Work }\SpecialCharTok{+}\NormalTok{ Price, }\AttributeTok{data =}\NormalTok{ data)}

\NormalTok{qda\_pred }\OtherTok{\textless{}{-}} \FunctionTok{predict}\NormalTok{(qda\_model, grid)}
\NormalTok{grid}\SpecialCharTok{$}\NormalTok{qda\_class }\OtherTok{\textless{}{-}}\NormalTok{ qda\_pred}\SpecialCharTok{$}\NormalTok{class}


\CommentTok{\# Plot QDA decision boundaries}
\NormalTok{qda\_plot }\OtherTok{\textless{}{-}} \FunctionTok{ggplot}\NormalTok{() }\SpecialCharTok{+} 
  \FunctionTok{geom\_point}\NormalTok{(}\AttributeTok{data =}\NormalTok{ grid, }\FunctionTok{aes}\NormalTok{(}\AttributeTok{x =}\NormalTok{ Work, }\AttributeTok{y =}\NormalTok{ Price, }\AttributeTok{color =}\NormalTok{ qda\_class), }\AttributeTok{alpha =} \FloatTok{0.3}\NormalTok{) }\SpecialCharTok{+} 
  \FunctionTok{geom\_point}\NormalTok{(}\AttributeTok{data =}\NormalTok{ data, }\FunctionTok{aes}\NormalTok{(}\AttributeTok{x =}\NormalTok{ Work, }\AttributeTok{y =}\NormalTok{ Price, }\AttributeTok{color =}\NormalTok{ SalaryCat)) }\SpecialCharTok{+}
  \FunctionTok{labs}\NormalTok{(}\AttributeTok{title =} \StringTok{"QDA: Work vs Price Decision Boundary"}\NormalTok{,}
       \AttributeTok{x =} \StringTok{"Work (Weighted Average of Working Hours)"}\NormalTok{,}
       \AttributeTok{y =} \StringTok{"Price (Index)"}\NormalTok{) }\SpecialCharTok{+}
  \FunctionTok{theme\_minimal}\NormalTok{()}

\NormalTok{qda\_plot}
\end{Highlighting}
\end{Shaded}

\includegraphics{day4_Final_files/figure-latex/unnamed-chunk-10-1.pdf}
\#\#\# LDA vs QDA

\paragraph{LDA}\label{lda}

The salary categories (35.1, 67.6{]} (green) and (67.6, 100{]} (blue)
are more clearly separated. The linear boundary works well in this case,
especially in the middle and upper categories. For example, the green
region is formed between about 1700 and 2100 work hours, while blue is
in the lower working hours and higher prices. However, the lower salary
category (2.6, 35.1{]} (red) tends to dominate more at the extreme
ranges of Work.

\paragraph{QDA}\label{qda}

The regions are similar to LDA, but QDA produces more curved and nuanced
boundaries, especially in areas where there might be overlaps between
salary categories. The flexibility of QDA allows for a more detailed
separation in these regions. For example, the red region expands
differently in QDA, accommodating some data points that were likely
misclassified in LDA due to its rigid linear constraints.

\subsubsection{Conclusion}\label{conclusion}

For this data, QDA fits better. We can see that it effectively captures
the non-linear relationships between Work and Price (this is especially
marked for our Blue class, which LDA ineffectively captured), providing
more detailed decision boundaries that better separate the salary
categories. LDA, while useful for linearly separable data, struggles
with the complexity of this dataset, leading to some misclassifications
and less distinct boundaries.

\subsection{1.E}\label{e}

\begin{itemize}
\tightlist
\item
  EDA provided insights into the relationships between Work, Salary, and
  Price, showing clear patterns across cities.
\item
  K-means clustering revealed three distinct groups of cities with
  different economic profiles based on their work hours and salary
  levels.
\item
  PCA reduced the dimensionality of the data and visualized how these
  cities are distributed across the two main sources of variation (Work
  and Salary/Price), confirming the clustering results.
\item
  LDA and QDA provided classification models for predicting salary
  categories based on Work and Price. QDA's quadratic boundaries offered
  superior classification accuracy, while LDA's linear approach
  struggled with the non-linear relationships in the data. QDA proved to
  be the most effective classification method for this dataset, while
  PCA and K-means clustering provided strong tools for understanding the
  structure of the data and identifying patterns among cities.
\end{itemize}

\section{Problem 2 (classification trees \& random
forests)}\label{problem-2-classification-trees-random-forests}

\subsection{Load the R data file}\label{load-the-r-data-file}

\begin{Shaded}
\begin{Highlighting}[]
\CommentTok{\# Load the data}
\FunctionTok{load}\NormalTok{(}\StringTok{"data/wine.RData"}\NormalTok{)}
\CommentTok{\# Check the structure of the data}
\FunctionTok{str}\NormalTok{(wine)}
\end{Highlighting}
\end{Shaded}

\begin{verbatim}
## 'data.frame':    178 obs. of  14 variables:
##  $ Type           : Factor w/ 3 levels "1","2","3": 1 1 1 1 1 1 1 1 1 1 ...
##  $ Alcohol        : num  14.2 13.2 13.2 14.4 13.2 ...
##  $ Malic          : num  1.71 1.78 2.36 1.95 2.59 1.76 1.87 2.15 1.64 1.35 ...
##  $ Ash            : num  2.43 2.14 2.67 2.5 2.87 2.45 2.45 2.61 2.17 2.27 ...
##  $ Alcalinity     : num  15.6 11.2 18.6 16.8 21 15.2 14.6 17.6 14 16 ...
##  $ Magnesium      : int  127 100 101 113 118 112 96 121 97 98 ...
##  $ Phenols        : num  2.8 2.65 2.8 3.85 2.8 3.27 2.5 2.6 2.8 2.98 ...
##  $ Flavanoids     : num  3.06 2.76 3.24 3.49 2.69 3.39 2.52 2.51 2.98 3.15 ...
##  $ Nonflavanoids  : num  0.28 0.26 0.3 0.24 0.39 0.34 0.3 0.31 0.29 0.22 ...
##  $ Proanthocyanins: num  2.29 1.28 2.81 2.18 1.82 1.97 1.98 1.25 1.98 1.85 ...
##  $ Color          : num  5.64 4.38 5.68 7.8 4.32 6.75 5.25 5.05 5.2 7.22 ...
##  $ Hue            : num  1.04 1.05 1.03 0.86 1.04 1.05 1.02 1.06 1.08 1.01 ...
##  $ Dilution       : num  3.92 3.4 3.17 3.45 2.93 2.85 3.58 3.58 2.85 3.55 ...
##  $ Proline        : int  1065 1050 1185 1480 735 1450 1290 1295 1045 1045 ...
\end{verbatim}

\subsection{2.A}\label{a-1}

\begin{Shaded}
\begin{Highlighting}[]
\FunctionTok{set.seed}\NormalTok{(}\DecValTok{111}\NormalTok{)}
\CommentTok{\# Split the data into training and testing sets}
\NormalTok{train\_index }\OtherTok{\textless{}{-}} \FunctionTok{sample}\NormalTok{(}\DecValTok{1}\SpecialCharTok{:}\FunctionTok{nrow}\NormalTok{(wine), }\FloatTok{0.7} \SpecialCharTok{*} \FunctionTok{nrow}\NormalTok{(wine))}
\NormalTok{train\_data }\OtherTok{\textless{}{-}}\NormalTok{ wine[train\_index, ]}
\NormalTok{test\_data }\OtherTok{\textless{}{-}}\NormalTok{ wine[}\SpecialCharTok{{-}}\NormalTok{train\_index, ]}
\end{Highlighting}
\end{Shaded}

\subsection{classification trees}\label{classification-trees}

\begin{Shaded}
\begin{Highlighting}[]
\NormalTok{tree\_model }\OtherTok{\textless{}{-}} \FunctionTok{rpart}\NormalTok{(Type }\SpecialCharTok{\textasciitilde{}}\NormalTok{ ., }\AttributeTok{data =}\NormalTok{ train\_data, }\AttributeTok{method =} \StringTok{"class"}\NormalTok{)}
\NormalTok{tree\_pred }\OtherTok{\textless{}{-}} \FunctionTok{predict}\NormalTok{(tree\_model, test\_data, }\AttributeTok{type =} \StringTok{"class"}\NormalTok{)}
\NormalTok{confusion\_matrix\_tree }\OtherTok{\textless{}{-}} \FunctionTok{table}\NormalTok{(tree\_pred, test\_data}\SpecialCharTok{$}\NormalTok{Type)}
\NormalTok{confusion\_matrix\_tree}
\end{Highlighting}
\end{Shaded}

\begin{verbatim}
##          
## tree_pred  1  2  3
##         1 14  2  1
##         2  3 16  0
##         3  0  1 17
\end{verbatim}

\begin{Shaded}
\begin{Highlighting}[]
\NormalTok{tree\_accuracy }\OtherTok{\textless{}{-}} \FunctionTok{confusionMatrix}\NormalTok{(tree\_pred, test\_data}\SpecialCharTok{$}\NormalTok{Type)}\SpecialCharTok{$}\NormalTok{overall[}\StringTok{"Accuracy"}\NormalTok{]}
\FunctionTok{print}\NormalTok{(}\FunctionTok{paste}\NormalTok{(}\StringTok{"Classification Tree Accuracy:"}\NormalTok{, }\FunctionTok{round}\NormalTok{(tree\_accuracy }\SpecialCharTok{*} \DecValTok{100}\NormalTok{, }\DecValTok{2}\NormalTok{), }\StringTok{"\%"}\NormalTok{))}
\end{Highlighting}
\end{Shaded}

\begin{verbatim}
## [1] "Classification Tree Accuracy: 87.04 %"
\end{verbatim}

\begin{Shaded}
\begin{Highlighting}[]
\FunctionTok{rpart.plot}\NormalTok{(tree\_model, }\AttributeTok{main =} \StringTok{"Classification Tree for Wine Data"}\NormalTok{)}
\end{Highlighting}
\end{Shaded}

\includegraphics{day4_Final_files/figure-latex/unnamed-chunk-13-1.pdf}

\subsection{Random Forest}\label{random-forest}

\begin{Shaded}
\begin{Highlighting}[]
\NormalTok{rf\_model }\OtherTok{\textless{}{-}} \FunctionTok{randomForest}\NormalTok{(Type }\SpecialCharTok{\textasciitilde{}}\NormalTok{ ., }\AttributeTok{data =}\NormalTok{ train\_data, }\AttributeTok{ntree =} \DecValTok{500}\NormalTok{, }\AttributeTok{mtry =} \DecValTok{3}\NormalTok{, }\AttributeTok{importance =} \ConstantTok{TRUE}\NormalTok{)}

\CommentTok{\# Predict on the test set}
\NormalTok{rf\_pred }\OtherTok{\textless{}{-}} \FunctionTok{predict}\NormalTok{(rf\_model, test\_data)}

\CommentTok{\# eval}
\NormalTok{rf\_accuracy }\OtherTok{\textless{}{-}} \FunctionTok{confusionMatrix}\NormalTok{(rf\_pred, test\_data}\SpecialCharTok{$}\NormalTok{Type)}\SpecialCharTok{$}\NormalTok{overall[}\StringTok{"Accuracy"}\NormalTok{]}
\FunctionTok{print}\NormalTok{(}\FunctionTok{paste}\NormalTok{(}\StringTok{"Random Forest Accuracy:"}\NormalTok{, }\FunctionTok{round}\NormalTok{(rf\_accuracy }\SpecialCharTok{*} \DecValTok{100}\NormalTok{, }\DecValTok{2}\NormalTok{), }\StringTok{"\%"}\NormalTok{))}
\end{Highlighting}
\end{Shaded}

\begin{verbatim}
## [1] "Random Forest Accuracy: 98.15 %"
\end{verbatim}

\begin{Shaded}
\begin{Highlighting}[]
\FunctionTok{varImpPlot}\NormalTok{(rf\_model, }\AttributeTok{main =} \StringTok{"Variable Importance in Random Forest Model"}\NormalTok{)}
\end{Highlighting}
\end{Shaded}

\includegraphics{day4_Final_files/figure-latex/unnamed-chunk-14-1.pdf}

\subsection{Cross Validation}\label{cross-validation}

\begin{Shaded}
\begin{Highlighting}[]
\FunctionTok{set.seed}\NormalTok{(}\DecValTok{111}\NormalTok{)}
\NormalTok{cross\_validate }\OtherTok{\textless{}{-}} \ControlFlowTok{function}\NormalTok{(data, dependent\_var, }\AttributeTok{n\_folds =} \DecValTok{10}\NormalTok{, model\_type) \{}
  \CommentTok{\# Setup}
\NormalTok{  folds }\OtherTok{\textless{}{-}} \FunctionTok{sample}\NormalTok{(}\FunctionTok{rep}\NormalTok{(}\DecValTok{1}\SpecialCharTok{:}\NormalTok{n\_folds, }\AttributeTok{length.out =} \FunctionTok{nrow}\NormalTok{(data)))}
\NormalTok{  accuracy }\OtherTok{\textless{}{-}} \FunctionTok{numeric}\NormalTok{(n\_folds)}
\NormalTok{  precision }\OtherTok{\textless{}{-}} \FunctionTok{numeric}\NormalTok{(n\_folds)}
\NormalTok{  sensitivity }\OtherTok{\textless{}{-}} \FunctionTok{numeric}\NormalTok{(n\_folds)}


  \ControlFlowTok{for}\NormalTok{ (i }\ControlFlowTok{in} \DecValTok{1}\SpecialCharTok{:}\NormalTok{n\_folds) \{}
    \CommentTok{\# Setup test / training data}
\NormalTok{    train\_indices }\OtherTok{\textless{}{-}} \FunctionTok{which}\NormalTok{(folds }\SpecialCharTok{!=}\NormalTok{ i)}
\NormalTok{    test\_indices }\OtherTok{\textless{}{-}} \FunctionTok{which}\NormalTok{(folds }\SpecialCharTok{==}\NormalTok{ i)}

\NormalTok{    train\_data }\OtherTok{\textless{}{-}}\NormalTok{ data[train\_indices, ]}
\NormalTok{    test\_data }\OtherTok{\textless{}{-}}\NormalTok{ data[test\_indices, ]}

    \CommentTok{\# Create model}
    \ControlFlowTok{if}\NormalTok{ (model\_type }\SpecialCharTok{==} \StringTok{"ctree"}\NormalTok{) \{}
\NormalTok{      model }\OtherTok{\textless{}{-}} \FunctionTok{ctree}\NormalTok{(}\FunctionTok{as.formula}\NormalTok{(}\FunctionTok{paste}\NormalTok{(dependent\_var, }\StringTok{"\textasciitilde{} ."}\NormalTok{)), }\AttributeTok{data=}\NormalTok{train\_data)}
\NormalTok{    \}}
    \ControlFlowTok{else} \ControlFlowTok{if}\NormalTok{ (model\_type }\SpecialCharTok{==} \StringTok{"randomForest"}\NormalTok{)\{}
\NormalTok{      model }\OtherTok{\textless{}{-}} \FunctionTok{randomForest}\NormalTok{(}\FunctionTok{as.formula}\NormalTok{(}\FunctionTok{paste}\NormalTok{(dependent\_var, }\StringTok{"\textasciitilde{} ."}\NormalTok{)), }\AttributeTok{data=}\NormalTok{train\_data)}
\NormalTok{    \}}

    \CommentTok{\# Predict}
\NormalTok{    predictions }\OtherTok{\textless{}{-}} \FunctionTok{predict}\NormalTok{(model, test\_data)}
\NormalTok{    confusion\_matrix }\OtherTok{\textless{}{-}} \FunctionTok{table}\NormalTok{(predictions, test\_data[[dependent\_var]])}
\NormalTok{    accuracy[i] }\OtherTok{\textless{}{-}} \FunctionTok{sum}\NormalTok{(}\FunctionTok{diag}\NormalTok{(confusion\_matrix)) }\SpecialCharTok{/} \FunctionTok{sum}\NormalTok{(confusion\_matrix)}
\NormalTok{    precision[i] }\OtherTok{\textless{}{-}} \FunctionTok{mean}\NormalTok{(}\FunctionTok{diag}\NormalTok{(confusion\_matrix) }\SpecialCharTok{/} \FunctionTok{colSums}\NormalTok{(confusion\_matrix), }\AttributeTok{na.rm =} \ConstantTok{TRUE}\NormalTok{)}
\NormalTok{    sensitivity[i] }\OtherTok{\textless{}{-}} \FunctionTok{mean}\NormalTok{(}\FunctionTok{diag}\NormalTok{(confusion\_matrix) }\SpecialCharTok{/} \FunctionTok{rowSums}\NormalTok{(confusion\_matrix), }\AttributeTok{na.rm =} \ConstantTok{TRUE}\NormalTok{)}
\NormalTok{  \}}

  \CommentTok{\# Print results}
  \FunctionTok{list}\NormalTok{ (}
    \AttributeTok{accuracy =} \FunctionTok{mean}\NormalTok{(accuracy),}
    \AttributeTok{precision =} \FunctionTok{mean}\NormalTok{(precision),}
    \AttributeTok{sensitivity =} \FunctionTok{mean}\NormalTok{(sensitivity)}
\NormalTok{  )}
\NormalTok{\}}

\NormalTok{results\_ctree }\OtherTok{\textless{}{-}} \FunctionTok{cross\_validate}\NormalTok{(wine, }\StringTok{"Type"}\NormalTok{, }\AttributeTok{model\_type =} \StringTok{"ctree"}\NormalTok{)}
\NormalTok{results\_random\_forest }\OtherTok{\textless{}{-}} \FunctionTok{cross\_validate}\NormalTok{(wine, }\StringTok{"Type"}\NormalTok{, }\AttributeTok{model\_type =} \StringTok{"randomForest"}\NormalTok{)}

\FunctionTok{list}\NormalTok{(}
  \AttributeTok{results\_ctree =}\NormalTok{ results\_ctree,}
  \AttributeTok{results\_random\_forest =}\NormalTok{ results\_random\_forest}
\NormalTok{)}
\end{Highlighting}
\end{Shaded}

\begin{verbatim}
## $results_ctree
## $results_ctree$accuracy
## [1] 0.8986928
## 
## $results_ctree$precision
## [1] 0.8890873
## 
## $results_ctree$sensitivity
## [1] 0.893955
## 
## 
## $results_random_forest
## $results_random_forest$accuracy
## [1] 0.9826797
## 
## $results_random_forest$precision
## [1] 0.9896296
## 
## $results_random_forest$sensitivity
## [1] 0.9811111
\end{verbatim}

\subsection{Analysis}\label{analysis}

We obtain higher accuracy with random forest model compared to decision
tree. This is because random forest tends to be more robust by adding
more variability to the resulted trees by not only randomly selecting
observations (bootstraping), but also randomly including the variables
in the decision tress.

\end{document}
